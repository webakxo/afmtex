\documentclass[12pt,a4paper,hidelinks,oneside]{book} 












\usepackage{fontspec}
\usepackage{xunicode}
\usepackage{polyglossia}
\setmainlanguage{french}

\usepackage{graphicx}
\usepackage[export]{adjustbox}
\usepackage{float}

\usepackage{amsmath}
\usepackage{amssymb}
\usepackage{siunitx}
\usepackage[none]{hyphenat} %prevent words from breaking 
\usepackage[driverfallback=hypertex]{hyperref}
\gappto\captionsfrench{\renewcommand{\contentsname}{Sommaire}}
\title{La Modulation}
\author{fill page \\
	\and 
	fill page\\ 
}
\date{}

\begin{document}
\maketitle
%\pagenumbering{gobble}
\tableofcontents
% Pourquoi es-tu là ? (space)
%Je suis content ! (space)
%Mon nom est : Idriss (space)
%bla bla ; bla bla (space)
%[10:47 PM, 4/5/2020] Idriss Al Idrissi: Ceci est la fin d’une phrase. (no space)
%Ceci est la fin d’une phrase. (no space)
%Ceci est la fin d’une proposition, (no space)
\chapter{introduction sur la modulation}
\chapter{la modulation d'amplitude}
\section{la communication en modulation d'amplitude}
Un système de communication transmet à travers un canal des inormations de la source vers un utilisateur :
\begin{itemize}
	\item La source fournit l'information sous la forme d'une signal analogique ou numèrique (la souce peut peut être un microphone , un lecteur de disque
	capteur , données numériques $\ldots$ ) ;
	\item L'émetteur inscrit cette information sur l'amplitude d'une porteuse sinusoïdale de fréquence $f_{0}$ : c'est la modulation d'amplitude ( les informations sont inscrite dans l'amplitude de la porteuse);
	\item L'antenne d'émission transforme ce signal électrique en onde électromagnétique . l'antenne de reception effectue l'opération inverse;
	\item L'antenne de reception transforme l'onde électromagnetique en signal électrique et on retrouve a la base de l'antenne un signal qui est exactement 
	la même forme que le signal qui a été envoyé sur l'antenne d'émission, la différence est que, et a cause de la distance, le signal ici est beaucoup plus faible, et la valeur l'amplitude est dans les $\SI{}{\milli\volt}$ ou $\SI{}{\micro\volt}$ alors que dans l'amplitude de signal modulé est dans les $\SI{}{\volt}$.
	\item Le recepteur sélectionne dans la multitude des signaux captés par l'antenne, l'émission qui nous s'intéresse et il va extraire l'information c'est qu'on appele : la démodulation, et cette information peut être traduit par un haut (si l'information est un audio). 
\end{itemize}
\section{la modulation AM double bande avec porteuse}
\subsection{principe}
Pour Produire un signal modulé en amplitude, il faut :

\begin{itemize}
	\item Une porteuse sinusoïdale de fréquence $f_{0}$ .
	\item Une information BF qui peut être un signal audio, vidéo, analogique ou numérique.
\end{itemize}

Soit $p(\,t)\,=\cos2\pi f_{0}t$ la porteuse et $m(\,t)\ $ le message à transmettre. Le message AM DBAP s’écrit :la porteuse et $m(\,t)\ $ le message à transmettre. Le message AM DBAP s’écrit :
\begin{equation}
s(\,t)\,=(\,A+m(\,t)\,)\,\cos 2\pi f_{0}t
\end{equation}

Dans le cas d’un signal modulant sinusoïdal $m\left(t\right) = Am\cos 2\pi f_{m} t $, le signal AM DBAP devient :
\begin{equation}
\begin{aligned}
s(t) &=\left(A+A_{m} \cos 2 \pi f_{m} t\right) \cos 2 \pi f_{0} t \\
&=A\left(1+\frac{A_{m}}{A} \cos 2 \pi f_{m} t\right) \cos 2 \pi f_{0} t \\
&=A\left(1+k \cos 2 \pi f_{m} t\right) \cos 2 \pi f_{0} t
\end{aligned}
\end{equation}




Avec $k=\frac{Am}{A}$ 
: indice de modulation (ou taux de modulation) = rapport entre l’amplitude du signal
modulant et celle de la porteuse.
Pour un signal modulant quelconque, l’indice de modulation est défini par :
\begin{equation}
k=\frac{\left|m(t)\right|_{max}}{A}
\end{equation}

\subsection{représentation temporelle d’un signal DBAP}

Si $k \leq 1$, l’enveloppe du signal modulé $s\left(t\right)$ possède exactement la forme du signal modualnt. Si $k $>$ 1$, l’enveloppe du signal modulé ne correspond pas au signal modulant : le signal AM est surmodulé.
En pratique, on doit toujours avoir $ k \leq 1$.

Détermination de l’indice de modulation k à partir de la représentation temporelle du signal AM
DBAP :


 

\begin{align*}
&
\begin{cases}
s_{max} &=A\left(1+k\right)\\
s_{min} &=A\left(1-k\right)
\end{cases}       
\end{align*}
\begin{equation}
\Rightarrow  k=\frac{s_{max}-s_{min}}{s_{max}+s_{min} }
\end{equation}

\subsection{représentation spectrale d’un signal AM DBAP}

\begin{align*}
s\left(t\right) &= A \left(1+ k \cos 2 \pi f_{m} t\right) \cos 2 \pi f_{0} t \\
 &= A \cos 2 \pi f_{0} t + k A \cos 2 \pi f_{m} t \cos 2 \pi f_{0} t \\ 
 &= A \cos 2 \pi f_{0} t + \frac{ka}{2} \cos 2 \pi \left(f_{0}-f_{m}\right) t + \frac{ka}{2} \cos 2 \pi \left(f_{0} + f_{m}\right) t 
\end{align*}

Le spectre du signal AM DBAP possède donc une raie d’amplitude 𝐴 à la fréquence $f_{0}$ de la porteuse
et deux raies latérales d’amplitude $ka$
2
aux fréquences $f_{0}-f_{m}$ et $f_{0}+f_{m}$

\begin{itemize}
	\item la porteuse a une amplitude $A$ ;
	\item les raies latérales superieure et inférieure ont la même amplitude $\frac{ka}{2}$ ;
	\item l'encombrement spectral du signal AM est le double de la fréquence \mbox{BF :} $B=2f_{m}$.
	
\end{itemize}

%\begin{itemize}
	\noindent\textbf{Remarque} : la place ocupée par ce signal modulé va de $f_{0}-f_{m} \rightarrow f_{0}+f_{m}$, donc la bande ocupée est égale a $2$ fois la fréquence de signal modulant. si nous voulons produire un signal modulé en amplitude par un signal BF à $\SI{10}{\kilo\hertz}$, notre spectre en signal modulé aura un largeur de  $ 2\times3= \SI{6}{\kilo\hertz}$,le faite de manipuler l'amplitude de la sinusoïde m'élargit le spectre de signal. 
	
%\end{itemize}

Dans le pratique où \textbf{le signal modulant est quelconque} mais de spectre borné,\textit{John Renshaw Carson} à démontré en $1914$ que le spectre a une forme semblable.
%\textit{John Renshaw Carson} à démontré en $1914$
%Wrong, don’t talk about reserserch in the core text, put bibliography mark to the entry of the paper (with the author, date, publisher, city publishing, and issn number if any)
\subsection{puissance d’un signal AM DBAP}

Dans le cas d’un signal modulant sinusoïdal, le signal appliquée à l'entenne a $3$ composantes, donc la puissance totale dissipée dans l'antenne est la somme de 3 puissance :

\begin{align*}
&
\begin{cases}
P_{s} &=P_{porteuse}+ P_{BLI} + P_{BLS}\\
P_{BLI}  &= P_{BLS}
\end{cases}
%\intertext{d'où}
%P_{s} &=P_{porteuse} + 2\times P_{BL}     
\end{align*} \\
d'où :

\begin{equation}
\begin{aligned}
P_{s} &=P_{porteuse} + 2\times P_{BL}
\end{aligned}
\end{equation}


\begin{equation}
P_{s}=\frac{A^{2}}{2}+2 \times \frac{\left(\frac{k A}{2}\right)^{2}}{2}=\frac{A^{2}}{2}+\frac{k^{2} A^{2}}{4}=\left(1+\frac{k^{2}}{2}\right) \frac{A^{2}}{2}
\end{equation}

	\begin{equation}
	\begin{aligned}
	P_{s}=\left(1+\frac{k^{2}}{2}\right) P_{ {porteuse }}
	\end{aligned}
	\end{equation}

En général, le signal AM transmis ne doit pas être surmodulé :$k\leq1$ . Pour $k=1$ (valeur maximale),
on a :
\begin{equation}
P_{s}= \frac{3}{2}P_{porteuse} \Rightarrow P_{porteuse}=\frac{2}{3}P_{s}
\end{equation}
Donc seul un tiers (au maximum) de la puissance du signal AM contient l’information utile. C’est un
inconvinient de la modualtion AM DBAP : \textit{gaspillage de puissance}.

%\section{La Production d'un signal AM} 
\section{démodulation des signaux AM DBAP}
\subsection{démodulation cohérente}
Pour un signal $m\left(t\right)$ tel que $|m\left(t\right)|_{max}=1$,on a : 
\begin{displaymath}
\begin{aligned}
v(t) &=s(t) \cos 2 \pi f_{0} t \\
&=A(1+k \cdot m(t)) \cos ^{2} 2 \pi f_{0} t \\
&=\frac{A}{2}(1+k \cdot m(t))\left(1+\cos 4 \pi f_{0} t\right) \\
&=\frac{A}{2}+\frac{k \cdot A}{2} m(t)+\frac{A}{2} \cos 4 \pi f_{0} t+\frac{k \cdot A}{2} m(t) \cos 4 \pi f_{0} t
\end{aligned}
\end{displaymath}
Aprés filtrage et suppression de la composante continue $\frac{A}{2}$, on obtient le signal  :
\begin{equation}
\hat{m}\left(t\right)=\frac{k \cdot A}{2}m\left(t\right)
\end{equation}

La démodulation cohérente présente le problème de la synchronisation de la porteuse locale avec la
porteuse à l’émission. Une méthode de démodulation plus efficace est \textit{la détection d’enveloppe}.

\subsection{démodulation AM par détection d’enveloppe}
\subsubsection*{principe}

mesure de l’enveloppe du signal pour récupérer le signal modulant \mbox{$m\left(t\right)$ :}

Détecteur d’enveloppe :

%\begin{figure}[!]
%	\centering
%	\includegraphics[max size={\textwidth}{\textheight}]{out.jpg}
%	\caption{redresseur}
%	\label{fig:redresseur}
%\end{figure}

\subsubsection*{fonctionnement}

\section{la modulation AM double bande sans porteuse}

\label{la modulation AM double bande sans porteuse}

\subsection{principe}
La modulation d'amplitude avec porteuse n'est pas le seul type de modulation, nous avons développer par la suite d'autre types de modulation en particulier la modulation double bande sans porteuse.

La modulation d'amplitude double bande sans porteuse (double side band suppressed carrier) est utilisée dans les multiplexages stéréo.
Le principe consiste à multiplier le signal modulant $m(t)$ avec la porteuse $p(t)$.
Le signal modulé ne contient pas le signal de la porteuse : ceci permet d'éviter
de le retrouver lors de la démodulation.
Le signal AM modulé en amplitude Double Bande Sans Porteuse (DBSP) s’écrit :
\begin{equation}
s(t)=p(t) \times m(t)
\end{equation}

\subsection{Cas d’un signal modulant sinusoïdal}

On considère le cas simple d'un signal modulant sinusoïdal $m(t)=A \cdot \cos \left(2 \pi f_{m} t\right)$ avec $f_{m} \ll f_{0}$. Le signal AM
s’écrit alors :
\begin{equation}
s(t)=\cos \left(2 \pi f_{0} t\right) \cdot A \cos \left(2 \pi f_{m} t\right)
\end{equation}
\paragraph{Représentation temporelle}
\paragraph{Représentation spectrale}
Pour déterminer le spectre de $s(t)$, il faut le décomposer en une somme de signaux sinusoïdaux. On a :
\begin{equation}
s(t)=\cos 2 \pi f_{0} t \cdot A \cos 2 \pi f_{m} t=\frac{A}{2} \cos 2 \pi\left(f_{0}+f_{m}\right) t+\frac{A}{2} \cos 2 \pi\left(f_{0}-f_{m}\right)t
\end{equation}
Le spectre d’amplitude du signal modulé $s(t)$ est donc constitué de deux raies symétriques situées aux
fréquences $f_{0}-f_{m}$ et $f_{0}+f_{m}$. De plus, il n’y a pas de composantes spectrales à la fréquence $f_{0}$ de la
porteuse. L’allure du spectre d’amplitude du signal modulé justifie l’appelation Double Bande Sans
Porteuse.

Le signal modulé est un signal à bande étroite, centré autour de la fréquence $f_{0}$ de la porteuse. Le but
de la modualtion est atteint : le signal BF est transformé en un signal HF.

\subsection{Cas d’un signal modualant quelconque}
\paragraph{Représentation temporelle}
\paragraph{Représentation spectrale }
Le spectre d’amplitude du signal AM DBSP avec un signal modulant quelconque est constitué de deux
bandes symétriques, centrées autour de $f_{0}$ : la bande latérale inférieure (BLI) et la bande latérale
supérieure (BLS).

L’occupation spectrale du signal AM DBSP est :
\begin{equation}
B_{s}=2\times B_{m}
\end{equation}

\noindent\textbf{remarque} : La transmission d’un signal en modulation AM DBSP nécessite donc une largeur de bande double de
celle du signal modulant.

\section{Démodulation des signaux AM DBSP}
\subsection{principe}
\subsection{Représentation spectrale}


\chapter{la modulation de fréquence} 
\section{introduction}
La modulation de fréquence est une forme de modulation angulaire, où la fréquence
varie en accordance avec le signal d'entré. ici, l'angle fait référence à la fréquence
angulaire $\omega$. la fréquence angulaire $\omega$ (ou vitesse de rotation) est est une grandeur qui représente
le rapport d'un angle de rotation au temps.
\begin{equation}
\omega=2 \pi f=\mathrm{d} \theta / \mathrm{d} t \label{3.1}
\end{equation}
où,
\begin{itemize}
	\item $\omega$ la fréquence angulaire
	\item f la fréquence en \si{\hertz}
	\item $\theta$ la phase 
\end{itemize}

Remarquons que dans l'équation \ref{3.1}, la fréquence angulaire et grand par rapport a la fréquence f par un facteur de $2\pi$. On integrons l'équation \ref{3.1}, On obtient:

\begin{equation}
\theta_{i}(t)=2 \pi \int_{0}^{t} f_{i} d t \label{3.2}
\end{equation}

\begin{itemize}
	\item $\theta_{i}$= la phase instantanée et
	\item $f_{i}$= la fréquence instantanée
\end{itemize}

Dans la suite, cette forme sera la base de notre dérivation de la modulation de fréquence.
\section{principe}
Pour produire un signal modulé, il faut :
\begin{itemize}
	\item une porteuse sinusuidale $\mathrm{e_0}(\mathrm{t})$ :
	\begin{equation}
	e_{0}(t)=E \cos (\omega t+\varphi) 
	\end{equation}
	\item une information basse-fréquence $\mathrm{s}(\mathrm{t})$
\end{itemize}

\section{production d'un signal FM}
\subsection{oscillateur contrôlé en tension VCO}
\subsubsection{définition}
L'oscillateur contrôlé en tension (Voltage controlled oscillator / VCO) est un dispositif électronique qui génère un signal dont la fréquence fluctue par rapport à la tension d'entrée.
\subsubsection{principe de fonctionnement :}
C'est un montage typiquement utilisé dans les PLL (Phase-Locked Loop, boucle à verrouillage de phase). Il permet ainsi de gérer la fréquence de sortie de la boucle de verrouillage, asservie avec la fréquence d'entrée.
\subsubsection{types :}
\begin{itemize}
	\item Le multivibrateur.
	\item L'oscillateur en anneau.
	\item Les oscillateurs RC.
	\item L'oscillateur à résonateur LC.
	\item Les oscillateurs basse fréquence fabriqués autour d'un NE555.
\end{itemize}

\subsection{production}
Pour émettre en modulation de fréquence il faut produire un signal a l'aide d'un un oscillateur commandé en tension (VCO), ce signal
doit etre sinusoïdal d’amplitude constante E et de fréquence variable, plus de ça on fixe le point de fonctionnement avec une polarisation
continue. finalement on fait varier la fréquence en superposant le signal $s\left(t\right)$ :
\begin{equation}
f(t)=f_{0}+k \times s(t) \label{f ins}
\end{equation}

à travers l'équation \ref{f ins} on peut passer à la pulsation instantannée en multipliant par $2\pi$ puis à la phase en integrant :
\begin{equation}
\begin{aligned}
&\omega(t)=2 \pi f(t)=\omega_{0}+2 \pi k\times s(t)\\
&\theta(t)=\int \omega(t) d t=\omega_{0} t+2 \pi k \int s(t)d t
\end{aligned}
\end{equation}
l'expression de la signal sinusoïdal d'amplitude E et de fréquence f(t) modulé en fréquence est : 
\begin{equation}
e(t)=E \cos (\theta(t))=E \cos \left(\omega_{0} t+2 \pi k \int s(t) d t\right)
\end{equation}
On va s'intéresser au cas d'un signal modulant sinusoïdal, $s(t)=\operatorname{acos}(\Omega t)$, donc l'expression de la signal porteuse sera :
\begin{equation}
\begin{aligned}
&e(t)=E \cos (\theta(t))=E \cos \left(\omega_{0} t+2 \pi k \int s(t) d t\right)=E \cos \left(\omega o t+\frac{2 \pi k a}{\Omega} \sin (\Omega t)\right)\\
&\text { d'où: } \quad e(t)=E \cos \left(\omega_{0} t+\frac{k a}{F} \sin (\Omega t)\right)
\end{aligned}
\end{equation}

%ici las figures porteuse avec modulant
\subsection{excursion en fréquence}
On appelle excursion de fréquence $\Delta f$, la variation instantanée de la fréquence de la porteuse par rapport à la fréquence de celle-ci non modulée. Cette variation est symétrique par rapport à la fréquence de la porteuse non modulée et se note $\pm \Delta f$.

quand le signal modulant s(t) évolue dans la plage $-{S_{max}}$, $+{S_{max}}$, la fréquence varie entre deux valeurs ${f_{min}}$ et ${f_{max}}$ :

${f_{min}}=f_{0}-k{S_{max}}$  et  ${f_{max}}=f_{0}+k{S_{max}}$

\subsection{Indice de modulation}

%begin indice de modulation 
Prenons le cas d'un signal modulant sinusoïdal de fréquence F, s(t) :
\begin{equation}
s(t)=a\cos(\Omega t)
\end{equation}
On définit l'indice de modulation $\beta$ par :
\begin{equation}
 \beta=\frac{\Delta f}{F} \label{indi beta}
\end{equation}

D'aprés cette équation \ref{indi beta}, on voit que l'indice de modulation $\beta$ augmente avec l'excursion en fréquence $\Delta f$.
\section{modulation de fréquence a bande étroite}























%conclusion
%Lors de sa découverte, la modulation de fréquence a fait l’objet de nombreux travaux de
%recherche (notamment en mathématiques) avant d’arriver à la conclusion que son occupation
%spectrale pour un même modulant était plus importante qu’en modulation d’amplitude %et d’en
%déduire qu’il fallait la ranger aux oubliettes. Il a fallu attendre les travaux d’Armstrong (1936)
%pour mettre en évidence les avantages de la modulation de fréquence, notamment en termes
%de comportement vis à vis des perturbations. Au prix d’une largeur de bande plus importante
%qu’en modulation d’amplitude, la modulation de fréquence améliore la qualité des
%transmissions en présence de bruit et rend possible la transmission d’un signal %musical de
%qualité. De plus, en modulation de fréquence, toute l’énergie émise contient de l’information
%et sa robustesse aux non linéarités rend possible l’utilisation d’amplificateurs en classe C
%possédant un bon rendement. Ces atouts en font une modulation particulièrement adaptée aux
%moyens de communication portables. 
 





\end{document}

